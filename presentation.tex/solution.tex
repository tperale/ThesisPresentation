\begin{frame}{Multi-hop}
\framesubtitle{What ?}
\begin{columns}
\begin{column}{0.6\textwidth}
\begin{itemize}
    \item Message relayed to neighbor
    \only<2->{ \item Range increased }
    \only<3->{ \item Adapt to topology }
    \only<4->{ \item Bi-directional }
    \only<5->{ 
        % \item Power efficient for border nodes
        \item Overall higher energy consumption 
    } 
\end{itemize}
\end{column}
\begin{column}{0.4\textwidth}
\begin{figure}[H]
    \centering
    \scalebox{0.8}{
    \begin{tikzpicture}[auto, thick]
      % Place super peers and connect them
      \foreach \place/\name in {{(0,0)/a}}
        \node[gateways] (\name) at \place {};
      \node[motes] (a4) at (-1.7, 1) {};
      \node[motes] (a5) at (-2.8, 1.3) {};
      \node[motes] (a7) at (-2.5, -0.5) {};
      \node[motes] (a8) at (-1.8, -1) {};
     \foreach \pos/\i in {below left of/1, below of/2, left of/3}
        \node[motes, \pos =a ] (a\i) {};
      \foreach \speer/\peer in {a/a1,a/a2,a/a3,a7/a3,a8/a3}
        \path[dotted] (\speer) edge (\peer);
      \only<1-2>{ 
        \node[motes, above left of=a ] (a6) {};
        \path[dotted,->] (a5) edge (a4);
        \path[dotted,->] (a4) edge (a6);
        \path[dotted,->] (a6) edge (a);
      }
      \only<3>{ 
        \path[dotted,->] (a5) edge (a4);
        \path[dotted,->] (a4) edge (a3);
        \path[dotted,->] (a3) edge (a);
      }
      \only<4->{ 
        \path[dotted,->] (a4) edge (a5);
        \path[dotted,->] (a3) edge (a4);
        \path[dotted,->] (a) edge (a3);
      }
      \only<2>{
        \draw[dotted,draw={black}] (-2.8,1.3) circle (1.3cm);
        \draw[dotted,draw={black}] (-1.65,1) circle (1.3cm);
      }
      % \draw[dotted,draw={black}] (-0.8,0.8) circle (1.4cm);
    \end{tikzpicture}
    }
    % \caption{Multi-hop Communication\label{fig:multihop}}
\end{figure}
\end{column}
\end{columns}
\end{frame}

% \begin{frame}{Multi-hop}
% \framesubtitle{LoRaWAN Extension}
% \begin{figure}[H]
%     \centering
%     \includegraphics[width=0.98\textwidth]{presentation.tex/fig/lorawanextension.png}
%     \caption{LoRaWAN Extension\footnotemark}
% \end{figure}

% \footcitetext{DIAS2018424}
% \end{frame}

% \begin{frame}{Multi-hop}
% \framesubtitle{Linear Sensor Network}
% \begin{columns}
% \begin{column}{0.5\textwidth}
% \begin{figure}[H]
%     \centering
%     \includegraphics[width=1\textwidth]{presentation.tex/fig/lsnlora.jpg}
%     \caption{Underground tunnel installation\footnotemark}
% \end{figure}
% \end{column}
% \begin{column}{0.5\textwidth}
% \begin{figure}[H]
%     \centering
%     \includegraphics[width=0.98\textwidth]{presentation.tex/fig/lsnlora2.jpg}
%     \caption{Protocol\footnotemark}
% \end{figure}
% \end{column}
% \end{columns}
% \footcitetext{Abrardo_2019}
% \end{frame}



% \begin{frame}{Multi-hop}
% \framesubtitle{LoRaBlink}
% \begin{columns}
% \begin{column}{0.3\textwidth}
% \begin{figure}[H]
%     \centering
%     \includegraphics[width=1\textwidth]{presentation.tex/fig/lorablink2.png}
%     \caption{Range\footnotemark}
% \end{figure}
% \end{column}
% \begin{column}{0.7\textwidth}
% \begin{figure}[H]
%     \centering
%     \includegraphics[width=0.98\textwidth]{presentation.tex/fig/lorablink.png}
%     \caption{LoRaBlink protocol\footnotemark}
% \end{figure}
% \end{column}
% \end{columns}
% \footcitetext{lorablink}
% \end{frame}

% \begin{frame}{Multi-hop}
% \framesubtitle{LoRa Multi-hop Network}
% % Sum up of multi-hop LoRa
% % What related work achieved
% \begin{itemize}
%     \item Cope with collision through
%     \begin{itemize}
%         \item Node coordination
%         \item Clock synchronization
%     \end{itemize}
%     \item Cope with bordering nodes through
%     \begin{itemize}
%         \item Increased range
%     \end{itemize}
%     \item Linear Sensor Network
%     % \item Interference resilience
% \end{itemize}
% \end{frame}

% \begin{frame}{Multi-hop}
% \framesubtitle{LoRa Multi-hop Network}
% % My work what it will achieve based on that
% \begin{itemize}
%     \item Low-power
%     \begin{itemize}
%         \item Time-Division
%         \item Synchronization
%         \item Scheduling
%     \end{itemize}
%     \item Reliable
%     \begin{itemize}
%         \item No collision
%         \item Interference resistance
%         \item Frequency-Division
%     \end{itemize}
%     \item Bi-directional
% \end{itemize}
% \end{frame}

\begin{frame}{Multi-hop}
\framesubtitle{Solving problem 1}
\begin{figure}[H]
    \centering
    \scalebox{1}{%
  \begin{tikzpicture}[]

  \begin{scope}[
      xshift=-0.2cm,
      asn/.style={black!70, minimum width=2cm},
      timeslot/.style={draw, rectangle, minimum width=2cm, minimum height=1cm},
      arr/.style={help lines,black!70,<->},
  ]
    \foreach \i in {0,...,4} {
      \node (ts\i) [asn] at (2*\i, 1) {$TS_{\i}$};
    }
    \node (tss0) [timeslot] at (0, 0) {\tiny Shared};
    \node (tss1) [timeslot] at (2, 0) {\tiny B $\rightarrow$ A};
    \node (tss2) [timeslot] at (4, 0) {\tiny C $\rightarrow$ B};
    \node (tss3) [timeslot] at (6, 0) {\tiny C $\rightarrow$ A};
    \node (tss4) [timeslot] at (8, 0) {\tiny A $\rightarrow$ C};
    \path[->] (-1.2,-1) edge (9.5,-1);
    \node [] at (10, -1.2) {time};
  \end{scope}
  \begin{scope}[xshift=3.8cm,yshift=-2cm,->,>=stealth',shorten >=1pt,auto,node distance=2.5cm]
    \tikzstyle{every state}=[thick,draw=gray!50,fill=gray!20,draw=none,text=black]

    \node[state]         (A) [] {A};
    \node[state]         (B) [below of=A]       {B};
    \node[state]         (C) [right of=B]       {C};

    \path (A) edge [bend left] node {$TS_4$} (C)
          (B) edge [bend left] node {$TS_1$} (A)
          (C) edge [bend left] node[above right] {$TS_3$} (A)
              edge [bend left] node {$TS_2$} (B);
  \end{scope}
\end{tikzpicture}
}
\end{figure}


\end{frame}

\begin{frame}{Multi-hop}
\framesubtitle{Solving problem 2 and 3}
\begin{columns}
\begin{column}{0.5\textwidth}

\begin{figure}[H]
    \centering
    \def\angle{0}
    \def\radius{3}
    \resizebox{4cm}{4cm}{%
    \begin{tikzpicture}[nodes = {font=\sffamily}]
      \foreach \color in {
            yellow,
            red,
            yellow,
            white,
            red,
            yellow,
            white,
            yellow,
            white,
            white,
            red,
            red,
        } {
        \ifx\color\empty\else
            \draw[fill={\color!50},draw={\color}] (0,0) -- (\angle:\radius)
              arc (\angle:\angle+30:\radius) -- cycle;
            \pgfmathparse{\angle+30}
            \xdef\angle{\pgfmathresult}
        \fi
        };
        \xdef\radius{2.5}
        \foreach \color in {
            yellow,
            red,
            yellow,
            yellow,
            red,
            white,
            yellow,
            red,
            white,
            white,
            white,
            white,
        } {
        \ifx\color\empty\else
            \draw[fill={\color!50},draw={\color}] (0,0) -- (\angle:\radius)
              arc (\angle:\angle+30:\radius) -- cycle;
            \pgfmathparse{\angle+30}
            \xdef\angle{\pgfmathresult}
        \fi
        };
        \xdef\radius{2}
        \foreach \color in {
            yellow,
            red,
            green,
            red,
            yellow,
            white,
            yellow,
            white,
            white,
            white,
            white,
            white,
        } {
        \ifx\color\empty\else
            \draw[fill={\color!50},draw={\color}] (0,0) -- (\angle:\radius)
              arc (\angle:\angle+30:\radius) -- cycle;
            \pgfmathparse{\angle+30}
            \xdef\angle{\pgfmathresult}
        \fi
        };
        \xdef\radius{1.5}
        \foreach \color in {
            yellow,
            red,
            yellow,
            yellow,
            yellow,
            green,
            yellow,
            white,
            red,
            white,
            white,
            white,
        } {
        \ifx\color\empty\else
            \draw[fill={\color!50},draw={\color}] (0,0) -- (\angle:\radius)
              arc (\angle:\angle+30:\radius) -- cycle;
            \pgfmathparse{\angle+30}
            \xdef\angle{\pgfmathresult}
        \fi
        };
        \xdef\radius{1}
        \foreach \color in {
            green,
            green,
            green,
            yellow,
            yellow,
            yellow,
            white,
            white,
            yellow,
            red,
            yellow,
            red,
        } {
        \ifx\color\empty\else
            \draw[fill={\color!50},draw={\color}] (0,0) -- (\angle:\radius)
              arc (\angle:\angle+30:\radius) -- cycle;
            \pgfmathparse{\angle+30}
            \xdef\angle{\pgfmathresult}
        \fi
        };
        \xdef\radius{0.5}
        \foreach \color in {
            green,
            green,
            yellow,
            yellow,
            green,
            green,
            yellow,
            red,
            yellow,
            green,
            green,
            green,
        } {
        \ifx\color\empty\else
            \draw[fill={\color!50},dotted,draw={\color}] (0,0) -- (\angle:\radius)
              arc (\angle:\angle+30:\radius) -- cycle;
            \pgfmathparse{\angle+30}
            \xdef\angle{\pgfmathresult}
        \fi
        };
        % \path[thick, dotted] (-0.7, -1.1) edge (0,0);
        % \node[motes] () at (-0.7, -1.1) {};
        \path[thick, dotted] (-1.5, 2.2) edge (-0.5,1.2);
        \path[thick, dotted] (0, 0) edge (-0.5,1.2);
        \node[motes] () at (-1.5, 2.2) {};
        \node[motes] () at (-0.5, 1.2) {};
        \path[thick, dotted] (2.5, 2.3) edge (0.5,1.5);
        \path[thick, dotted] (0.5, 1.3) edge (0,0);
        \node[motes] () at (2.5, 2.3) {};
        \node[motes] () at (0.5, 1.3) {};

        \node[motes] () at (2.2, -2.1) {};
        \foreach \place/\name in {{(0,0)/a}}
            \node[gateways] (\name) at \place {};
    \end{tikzpicture}
    }
\end{figure}
\end{column}
\begin{column}{0.5\textwidth}
\begin{figure}[H]
    \centering
    \scalebox{0.6}{%
    \begin{tikzpicture}
    \path[] (-5,0) edge (3,0);
    \node[motes] (a4) at (-4.5, 0) {};
    \node[motes] (a5) at (-2.5, 0) {};
    \node[motes] (a6) at (0, 0) {};
    \node[motes] (a7) at (2.5, 0) {};
    \path (a4) edge [thick,bend left,dotted,->] node {     } (a5);
    \path (a5) edge [thick,bend left,dotted,->] node {     } (a6);
    \path (a6) edge [thick,bend left,dotted,->] node {     } (a7);
    % \node[gateways] (a) at (0,0) {};
    % \draw[dotted,draw={black}] (0,0) circle (3cm);
    \end{tikzpicture}
    }
\end{figure}
\end{column}
\end{columns}
\end{frame}
