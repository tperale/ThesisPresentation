\begin{frame}{Introduction}
\framesubtitle{IoT}
\begin{figure}
\centering
\includegraphics[width=0.4\textwidth]{presentation.tex/fig/iot.png}
\end{figure}
\end{frame}

\begin{frame}{Introduction}
\framesubtitle{Low Power Wide Area Network (LPWAN)}

\resizebox{9.8cm}{5.1cm}{%
  \begin{tikzpicture}
    \draw[->,thick] (-0.1,0)--(12,0) node[right]{Range};
    \draw[->,thick] (0,-0.1)--(0,8) node[above]{Data Rate};
    \node[] at (1, -0.5) {10m};
    \draw[] (1,-0.1)--(1,0.1);
    \node[] at (3, -0.5) {100m};
    \draw[] (3,-0.1)--(3,0.1);
    \node[] at (5, -0.5) {1km};
    \draw[] (5,-0.1)--(5,0.1);
    \node[] at (7, -0.5) {10km};
    \draw[] (7,-0.1)--(7,0.1);
    \node[] at (9, -0.5) {100km};
    \draw[] (9,-0.1)--(9,0.1);

    \node[] at (-1.2, 1) {1 kbit/sec};
    \draw[] (-0.1,1)--(0.1,1);
    \node[] at (-1.2, 3) {1 Mbit/sec};
    \draw[] (-0.1,3)--(0.1,3);
    \node[] at (-1.2, 5) {100 Mbit/sec};
    \draw[] (-0.1,5)--(0.1,5);
    \node[] at (-1.2, 7) {1 Gbit/sec};
    \draw[] (-0.1,7)--(0.1,7);

    \node[draw,fill=white] at (8,1.5) (lora) {LoRa};
    \node[draw,fill=white] at (8.7,0.8) (sigfox) {SigFox};
    \node[draw,fill=white] at (7.4,2.2) (nb) {NB-IOT};
    \begin{scope}[on background layer]
      \node[draw,fill=blue!30,fit=(lora) (sigfox) (nb), label=above:{LPWAN}] {};
    \end{scope}

    \node[draw,fill=white] at (2.2,3.5) (bluetooth) {Bluetooth};
    \node[draw,fill=white] at (2.7,2.5) (zigbee) {ZigBee};
    \node[draw,fill=white] at (2.0,1.8) (ble) {BLE};
    \begin{scope}[on background layer]
      \node[draw,fill=blue!10,fit=(bluetooth) (zigbee) (ble), label=above:{Short-range}] {};
    \end{scope}

    \node[draw,fill=white] at (3,5) (wifi) {WiFi};
    \begin{scope}[on background layer]
      \node[draw,fill=blue!50,fit=(wifi)] {};
    \end{scope}

    \node[draw,fill=white] at (6.8,5) (lte) {LTE};
    \node[draw,fill=white] at (5.8,6) (5g) {5G};
    \begin{scope}[on background layer]
      \node[draw,fill=blue!60,fit=(lte) (5g), label=above:{Cellular}] {};
    \end{scope}

    \node[] at (6.9,8.2) (leg5) {\small Low};
    \node[draw,fill=blue!10] at (7.5,8.2) (leg1) {};
    \node[draw,fill=blue!30] at (7.8,8.2) (leg2) {};
    \node[draw,fill=blue!50] at (8.1,8.2) (leg3) {};
    \node[draw,fill=blue!60] at (8.4,8.2) (leg4) {};
, visible on=<4->    \node[] at (10.7,8.2) (leg5) {\small High Power Consumption};
\end{tikzpicture}
}
\end{frame}

\begin{frame}{Introduction}
\framesubtitle{LoRa}
\begin{columns}
\begin{column}{0.5\textwidth}
\begin{itemize}
  \item Proprietary Modulation Technique
  \begin{itemize}
    \item Owned by Semtech
  \end{itemize}
  % \item Chirp Spread Spectrum (CSS)
  % \item Spreading Factors (SF)
  % \begin{itemize}
  %   \item $\nearrow$ SF, $\searrow$ Data-rate
  %   \item $\nearrow$ SF, $\nearrow$ Range
  % \end{itemize}
  \item Range of around ~15km
  \item Low-power transmission
  \item Very low data-rate
\end{itemize}
\end{column}
\begin{column}{0.5\textwidth}
\begin{figure}
\includegraphics[width=0.8\textwidth]{presentation.tex/fig/lora.jpg}
\end{figure}
\end{column}
\end{columns}
\end{frame}

\begin{frame}{Introduction}
\framesubtitle{LoRaWAN}
\begin{columns}
\begin{column}{0.5\textwidth}
% \begin{itemize}
%     \item Stars-of-stars topology
%     \begin{itemize}
%       \item End-nodes/Motes
%       \item Gateways
%       \item Central Server
%     \end{itemize}
%     \item Direct Link to gateway
%     \item Transmit as soon as possible
%     \begin{itemize}
%       \item Low channel occupation
%       \item Power efficient
%     \end{itemize}
% \end{itemize}
\begin{figure}[H]
    \centering
    \begin{tikzpicture}[auto, thick]
      % Place super peers and connect them
      \foreach \place/\name in {{(-1,-2)/a}, {(-0,1)/b}}
        \node[gateways] (\name) at \place {};
      \node[server] (d) at (1.5,-1.5) {};
      %
      \begin{scope}[on background layer]
      \foreach \source/\dest in {a/d, b/d}
        \path[] (\source) edge (\dest);
      \end{scope}

      % Place normal peers
      \foreach \pos/\i in {above left of/1, left of/2, below left of/3}
        \node[motes, \pos =b ] (b\i) {};
      \foreach \speer/\peer in {b/b1,b/b2,b/b3}
        \path[dotted] (\speer) edge (\peer);
      %
      \foreach \pos/\i in {below left of/1, below of/2, left of/3}
        \node[motes, \pos =a ] (a\i) {};
      \foreach \speer/\peer in {a/a1,a/a2,a/a3}
        \path[dotted] (\speer) edge (\peer);
      %
      \path[dotted] (a) edge (b3);
    \end{tikzpicture}
    \caption{Star topology\label{fig:startopology}}
\end{figure}

\end{column}
\begin{column}{0.5\textwidth}
\begin{figure}[H]
\centering
\scalebox{0.8}{%
\begin{tikzpicture}[->,>=stealth',shorten >=1pt,auto,node distance=1.6cm]
  \tikzstyle{comment}=[ right=2pt, font=\small, fill=white, text=black, draw=black, ]
  \tikzstyle{every state}=[rectangle,thick, draw=black,fill=gray!20,text=black, minimum width= 6cm, minimum height= 1.20cm ]
  \tikzstyle{smallstate}=[rectangle,thick, draw=black,fill=gray!10,text=black, minimum width= 4cm, minimum height= 1.20cm ]
  \node[state]         (AB)                 {Application};
  \node[state]         (C) [below of=AB]       {LoRaWAN};
  \node[state]         (D) [below of=C]       {LoRa};
  \node[comment]       at (C.north west) {MAC Layer};
  \node[comment]       at (D.north west) {Physical Layer};
\end{tikzpicture}
}
\end{figure}
\end{column}
\end{columns}
\end{frame}


